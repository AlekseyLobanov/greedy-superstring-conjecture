\section{Proof of Collapsing Conjecture for Strings of Length~3}
\label{subsec:scs3}
In Sections~\ref{subsec:scs3}--\ref{sec:proof2} we prove the Collapsing Conjecture for strings of length at most~$3$.

We will use the following notation. For a directed graph $G=(V,E)$, we say that $P$ is a weak path between vertices $u, v\in V$ if replacing all arcs of $G$ with undirected edges produces a graph where $P$ is an undirected path between $u$ and $v$.

The Collapsing Algorithm (CA) starts with $D$ equal to a doubled (arbitrary) solution, and the Greedy Hierarchical Algorithm (GHA) starts with an~empty solution. Then both algorithms consider all vertices of the hierarchical graph $HG=(V, E)$ in the same order (descending order of the levels, lexicographic order within a level). Let $s_0$ be the first vertex considered by CA and GHA ($s_0$ corresponds to an input string). For a~vertex $s$, by $V(s)$ we denote the set consisting of~$s$ and all the vertices that we processed before~$s$.
%of vertices considered by the algorithms no later than the vertex $s$ (including the vertex $s$: $s\in V(s)$).

Both algorithms continuously change their partial solutions. By $\cld{}$ and $\grd{}$ we denote the set of arcs included in the partial solutions $D$ of the algorithms CA and GHA right before they start examining the vertex $s$. Let $\clgraph = (V, \cld), \grgraph = (V, \grd)$, where $V$ is the set of vertices of the hierarchical graph $HG$. 

Finally, for CA we define the set $\cldr{s}$, consisting of arcs of $\cld{}$ which are incident to at least one vertex from $ V (s) \setminus \{s \} $ (recall that $\cld{}$ is the set of arcs strictly before the examination of the vertex $s$). For convenience, for GHA we define $\grdr{s}$ in the same way, but note that since GHA always adds arcs adjacent to the vertex it is currently considering, $\grdr{s}=\grd$.\footnote{On the other hand, $\cld$ does not necessarily equal $\cldr{s}$, because $\cld{}$ initially consists of arcs connecting all vertices from $ {\cal S} $ to $\varepsilon$.}

Let us sort all vertices of $V$ in the order in which the two algorithms process them, and let $s$ and $t$ be two consecutive vertices in this order. Now look at the two algorithms processing the vertex $s$. CA and GHA both only change arcs below $s$. Since $\cld(t)$ and $\grd(t)$ only include the arcs incident to vertices from $V(t)$, the following holds for both algorithms:
\begin{equation}
\label{eqn:stepcase}
    D(t) = D(s) \sqcup \bigsqcup_{i=1}^{a} (\pref(s), s) \sqcup \bigsqcup_{j=1}^{b} (s, \suff(s))
\end{equation}
for some non-negative integers $ a $ and~$ b $.

In Sections~\ref{sec:proof1}--\ref{sec:proof2} we will prove the following lemma.
\begin{lemma}
\label{lem:technical}
Let ${\cal S}$ be a set of strings of length at most $3$. For each vertex $ s $ at a level at least $1$, if the collapsing algorithm after its examination leaves $ a $ arcs $ (\pref (s), s) $ and $ b$ arcs $ (s, \suff(s) ) $, and the greedy algorithm leaves $ a_{gr} $ and $ b_{gr} $ arcs, respectively, then $ a = a_{gr}, b = b_{gr} $.
\end{lemma}

Now we are ready to complete the proof of Collapsing Conjecture for strings of length at most $3$. Note that this also implies that the Greedy Hierarchical Conjecture holds for the case of strings of length at most $3$.

\begin{theorem}
\label{thm:main}
For any set ${\cal S}$ of strings of length at most $3$ and for any superstring $C$ of ${\cal S}$, 
\begin{align*}
CA(HG({\cal S}), C \sqcup C) = GHA({\cal S}) \; .
\end{align*}
\end{theorem}

\begin{proof}
We will prove that
\begin{align*}
\forall s \in V\colon\, \cldr{s}=\grdr{s} \; .
\end{align*}
For $s=\varepsilon$, this statement implies the Collapsing Conjecture, as it asserts that the two algorithms end up with the same multiset of arcs. 

We prove this statement by induction on $s$, where $s$ starts with $s_0$ and goes through all the vertices of $HG$ in the same order as CA and GHA. The base case $s=s_0$ of the induction argument follows trivially from the definition: $\cldr{s_0}=\grdr{s_0}=\emptyset$. And the induction step is proven in Lemma~\ref{lem:technical}.
\end{proof}
\input proof1
\input proof2
