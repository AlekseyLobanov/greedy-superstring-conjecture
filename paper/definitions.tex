\section{Definitions}
\subsection{Shortest Common Superstring Problem}
\label{sec:def_scs}
For a string $s$, by $|s|$ we denote the length of $s$. 
For strings~$s$ and~$t$, by $\overlap(s,t)$
we denote the longest suffix of~$s$ that is also 
a~prefix of~$t$. By $\pref(s,t)$
we denote the first $|s|-|\overlap(s,t)|$ symbols of $s$.
Similarly, $\suff(s,t)$ is the last
$|t|-|\overlap(s,t)|$ symbols of~$t$. 
By $\pref(s)$ and $\suff(s)$ we denote, respectively,
the first and the last $|s|-1$ symbols of~$s$. See Figure~\ref{fig:overlap} for a~visual explanation. We denote the empty string by $\varepsilon$.


\begin{figure}[ht]
\begin{mypic}
\begin{scope}
%\draw[help lines,step=5mm] (0,0) grid (14,3);

\draw (0,2) rectangle (6,2.5);
\draw[step=5mm] (0,2) grid (6,2.5);
\node[left] at (0,2.25) {$s$};
\draw (3,0.5) rectangle (10,1);
\draw[step=5mm] (3,0.5) grid (10,1);
\node[right] at (10,0.75) {$t$};

\foreach \x in {3.25, 3.75, ..., 5.75}
  \draw[gray,thick] (\x,2) -- (\x,1);

\foreach \f/\t/\y/\lab in {0/3/1.5/{\pref(s,t)}, 
3/6/0/{\overlap(s,t)}, 6/10/0/{\suff(s,t)}, 0/5.5/3/\pref(s),
0.5/6/3.5/\suff(s)}
  \path (\f,\y) edge[<->] node[rectangle,inner sep=0.5mm,fill=white] {\strut $\lab$} (\t,\y);
  
\foreach \x/\a in {0/b, 0.5/a, 1/a, 1.5/c, 2/a, 2.5/b, 3/b, 3.5/c, 4/a, 4.5/a, 5/c, 5.5/b}
  \node at (\x+0.25,2.25) {\tt \a};
\foreach \x/\a in {3/b, 3.5/c, 4/a, 4.5/a, 5/c, 5.5/b, 6/a, 6.5/c, 7/a, 7.5/a, 8/a, 8.5/b, 9/c, 9.5/a}
  \node at (\x+0.25,0.75) {\tt \a};
\end{scope}
\end{mypic}
\caption{Pictorial explanations of $\pref$, $\suff$, and $\overlap$ functions.}
\label{fig:overlap}
\end{figure}


Throughout the paper by ${\cal S}=\{s_1, \dots, s_n\}$ we denote
the set of~$n$ input strings. We assume that no input string is a~substring of another (such a~substring can be removed from $\mathcal{S}$ in the preprocessing stage). Note that SCS is a~{\em permutation problem}: to find a~shortest string containing all $s_i$'s in a~{\em given order} one just
overlaps the strings in this order, see Figure~\ref{fig:permutation}. (This simple observation relates SCS to other permutation problems, including various versions of the Traveling Salesman Problem.) It will prove convenient to view the SCS problem as a~problem of finding an optimum permutation.
It should be noted at the same time that the correspondence between permutations and superstrings is not one-to-one: there are superstrings that do not correspond to any permutation. For example, the concatenation of input strings is clearly a~superstring, but it ignores the fact that neighbor strings may have non-trivial overlaps and for this reason may fail to correspond to a~permutation. Still, clearly, any {\em shortest} superstring corresponds to a~permutation of the input strtings.


\begin{figure}[ht]
\begin{mypic}
%\begin{scope}[scale=0.85]
%\draw[help lines] (0,0) grid (16,4);

\foreach \x in {0, 2.5, 5.5, 10, 12.5, 16}
  \draw[dashed,gray,thin] (\x,-0.5) -- (\x,4);

\foreach \x/\y/\len/\label in {0/3/5/s_{i_1}, 1/2.5/9/s_{i_2}, 2/2/16/s_{i_3}, 10.5/1/4/s_{i_{n-1}}, 12/0.5/8/s_{i_n}} {
  \draw (\x,\y) rectangle (\x+0.5*\len,\y+0.5);
  \node at (\x+0.25*\len,\y+0.25) {$\label$};
}

\foreach \f/\t/\label in {0/2.5/{s_{i_1}}, 2.5/5.5/{\suff(s_{i_1}, s_{i_2})}, 5.5/10/{\suff(s_{i_2}, s_{i_3})}, 12.5/16/{\suff(s_{i_{n-1}}, s_{i_n})}}
  \path (\f,0) edge[<->] node[rectangle,inner sep=0.5mm,fill=white] {\strut $\label$} (\t,0);

\node at (11,0) {$\dotsb$};
\node at (10.5,1.75) {$\dotsb$};
%\end{scope}
\end{mypic}
\caption{SCS is a~permutation problem. The length of a~superstring corresponding to a~permutation $(s_{i_1}, \dotsc, s_{i_n})$ is $|s_{i_1}|$ plus the sum of the lengths of suffixes of consecutive pairs of strings. It is also equal to $\sum_{i=1}^n |s_i|-\sum_{j=1}^{n-1}|\overlap(s_{i_j}, s_{i_{j+1}})|$.}
\label{fig:permutation}
\end{figure}


\subsection{Hierarchical Graph}
\label{sec:def_hier}
%\begin{definition}[hierarchical graph]
For a~set of strings~${\cal S}$, the~\emph{hierarchical graph} $HG=(V,E)$ is a~weighted directed graph with $V=\{v \colon \text{$v$ is a~substring of some $s \in {\cal S}$}\}$. For every $v \in V,\, v \neq \varepsilon$, the set of arcs~$E$ contains an {\em up-arc} $(\pref(v), v)$ of weight~1 and a {\em down-arc} $(v, \suff(v))$ of weight~0. The meaning of an up-arc is appending one symbol to the end of the current string (and that is why it has weight~1), whereas the meaning of a down-arc is cutting down one symbol from the beginning of the current string.
%\end{definition}
%
Figure~\ref{fig:hgex}(a) gives an example of the hierarchical graph and shows that the terminology of up- and down-arcs comes from placing all the strings of the same length at the same level, where the $i$-th level contains strings of length~$i$.  In all the figures in this paper, the input strings are shown in rectangles, while all other vertices are ellipses.

\newcommand{\we}[4]{
\begin{scope}[xshift=#1mm,yshift=#2mm]
\foreach \n/\x/\y in {aaa/0/3, cae/1/3, aec/3/3, eee/4/3}
  \node[inputvertex] (\n) at (\x,\y) {\tt \n};
%  
\foreach \n/\x/\y in {aa/0/2, ca/1/2, ae/2/2, ec/3/2, ee/4/2, a/1/1, c/2/1, e/3/1}
  \node[vertex] (\n) at (\x,\y) {\tt \n};
%
\node[vertex] (eps) at (2,0) {$\varepsilon$};
%
\foreach \f/\t/\a in {eps/e/10, e/eps/10, eps/c/10, c/eps/10, eps/a/10, a/eps/10, a/aa/10, aa/a/10, aa/aaa/10, aaa/aa/10, c/ca/0, ca/cae/0, cae/ae/0, ae/aec/0, aec/ec/0, ee/eee/10, eee/ee/10, e/ee/10, ee/e/10, ca/a/0, a/ae/0, ae/e/0, e/ec/0, ec/c/0}
  \path (\f) edge[hgedge,bend left=\a] (\t);
  
\node at (2,-1) {(#3)};

#4
\end{scope}
}

\begin{figure}[!ht]
\begin{mypic}
%\begin{scope}[scale=0.85]
\we{0}{0}{a}{}

\we{57}{0}{b}{
\foreach \f/\t/\a in {eps/a/10, a/ae/0, ae/aec/0, aec/ec/0, ec/c/0, c/ca/0, ca/a/0, a/eps/10}
  \path (\f) edge[hgedge,bend left=\a,draw=black,thick] (\t);
}

\we{114}{0}{c}{
\foreach \f/\t/\a in {eps/a/10, a/aa/10, aa/aaa/10, aaa/aa/10, aa/a/10, a/ae/0, ae/aec/0, aec/ec/0, ec/c/0, c/ca/0, ca/cae/0, cae/ae/0, ae/e/0, e/ee/10, ee/eee/10, eee/ee/10, ee/e/10, e/eps/10}
  \path (\f) edge[hgedge,bend left=\a,draw=black,thick] (\t);
}
%\end{scope}
\end{mypic}
\caption{(a)~Hierarchical graph for the~dataset $\mathcal{S}=\{{\tt aaa}, {\tt cae}, {\tt aec}, {\tt eee}\}$. (b)~The~walk $\varepsilon \to {\tt a} \to {\tt ae} \to {\tt aec} \to {\tt ec} \to {\tt c} \to {\tt ca} \to {\tt a} \to \varepsilon$ has length (or weight)~4 and spells the~string {\tt aeca} of length~4. (c)~An~optimal superstring for~$\mathcal{S}$ is {\tt aaaecaeee}. It has length~9, corresponds to the~permutation $({\tt aaa}, {\tt aec}, {\tt cae}, {\tt eee})$, and defines the~walk of length~9 shown in black.}
\label{fig:hgex}
\end{figure}

What we are looking for in this graph is a shortest
walk from $\varepsilon$ to $\varepsilon$ going
through all the nodes from~$\mathcal{S}$.
It is not difficult to see that the length of a~walk 
from $\varepsilon$ to $\varepsilon$ equals the 
length of the string spelled by this walk. 
This is just because each up-arc has 
weight~$1$ and adds one symbol to 
the current string. See Figure~\ref{fig:hgex}(b) for an~example.
%For example, in the graph of Fiure.~\ref{fig:hgexfirst} 
%a~walk $\varepsilon\to{\tt b}\to{\tt ba}
%\to{\tt bab}\to{\tt ab}\to{\tt abc}
%\to{\tt abca}\to{\tt bca}\to{\tt ca}\to{\tt a}
%\to\varepsilon$ has length $5$ and spells 
%a~string {\tt babca} 
%of length~$5$ in a~natural way.\todo{Update this when you update the figure.}

Hence, the SCS problem is equivalent to finding a~shortest closed walk from $\varepsilon$ to $\varepsilon$ that visits all nodes from~${\cal S}$. Note that a walk may contain repeated nodes and arcs. The multiset of arcs of such a~walk must be Eulerian (each vertex must have the same in- and out-degree, and the set of arcs must be connected). It will prove convenient to define an~{\em Eulerian solution} in a~hierarchical graph as an~Eulerian multiset of arcs~$D$ that goes through $\varepsilon$ and all nodes from~${\cal S}$. Given such a~solution~$D$, one can easily recover an Eulerian cycle (that might not be unique). This cycle spells a~superstring of~${\cal S}$ of the same length as~$D$. Figure~\ref{fig:hgex}(c) shows an~optimal Eulerian solution.

A solution to SCS defines a permutation $(s_{i_1}, \dotsc, s_{i_n})$ of the input strings, and this permutation naturally gives a~``zig-zag'' Eulerian solution in the hierarchical graph:
\begin{align}
\label{eq:zigzag}
\varepsilon \to s_{i_1} \to \overlap(s_{i_1}, s_{i_2}) \to s_{i_2} \to
\overlap(s_{i_2}, s_{i_3}) \to \dotsb 
%\to \overlap(s_{i_{n-1}}, s_{i_n}) 
\to s_{i_n} \to \varepsilon \, .
\end{align}
This Eulerian solution is shown schematically in Figure~\ref{fig:hgperm}(a). This schematic illustration is over simplified as the shown path usually has many self-intersections. Still, this point of view is helpful in understanding the algorithms presented later in the text. 
Figure~\ref{fig:hgperm}(b) shows an ``untangled'' optimal Eulerain solution from Figure~\ref{fig:hgex}(c): by contracting nodes with equal labels into the same node, one gets exactly the solution from Figure~\ref{fig:hgex}(c).

\begin{figure}[ht]
\begin{mypic}
\begin{scope}[yscale=0.8]
%\draw[help lines] (-7.5,0) grid (7.5,3);
\foreach \n/\x/\y in {aaa/0/3, cae/4/3, aec/2/3, eee/6/3}
  \node[inputvertex] (\n) at (\x,\y) {\tt \n};

\foreach \n/\t/\x/\y in {aa1/aa/-0.5/2, aa2/aa/0.5/2, ae1/ae/1.5/2, ec/ec/2.5/2, ca/ca/3.5/2, ae2/ae/4.5/2, ee1/ee/5.5/2, ee2/ee/6.5/2, a1/a/0/1, a2/a/1/1, c/c/3/1, e1/e/5/1, e2/e/6/1}
  \node[vertex] (\n) at (\x,\y) {\tt \t};

\node[vertex] (eps) at (3,0) {$\varepsilon$};

\foreach \f/\t in {aa1/aaa, aaa/aa2, ae1/aec, aec/ec, ca/cae, cae/ae2, ee1/eee, eee/ee2, a1/aa1, aa2/a2, a2/ae1, ec/c, c/ca, ae2/e1, e1/ee1, ee2/e2}
  \draw[->] (\f) -- (\t);
  
\path (eps) edge[->,out=180,in=-45] (a1);
\path (e2) edge[->,out=-135,in=0] (eps);


\foreach \n/\x in {1/-8, 2/-7, 3/-6, n/-3}
  \node[inputvertex] (\n) at (\x,3) {$s_{i_{\n}}$}; 
  
\node at (-4.5,3) {$\dotsb$};

\foreach \n/\x/\y in {12/-7.5/2, 23/-6.5/1, 34/-5.5/1.5, 67/-3.5/2}
  \node[vertex] (\n) at (\x,\y) {}; 
  
\foreach \f/\t in {1/12, 12/2, 2/23, 23/3, 3/34, 67/n}
  \draw[->,anypath] (\f) -- (\t);
  
\node[vertex] (eps) at (-5.5,0) {$\varepsilon$};
\path (eps) edge[anypath,->,out=180,in=-90] (1);
\path (n) edge[anypath,->,out=-90,in=0] (eps);

\node at (-5.5,-1) {(a)};
\node at (3,-1) {(b)};
\end{scope}
\end{mypic}
\caption{(a)~A~schematic illustration of a~normalized Eulerian solution.
(b)~Untangled optimal Eulerian solution from Figure~\ref{fig:hgex}(c).}
\label{fig:hgperm}
\end{figure}

Not every Eulerian solution in the hierarchical graph has a~nice zig-zag
structure described above. In the next section, we introduce a~normalization procedure (that we call collapsing)
that allows us to focus on nice Eulerian solutions only. 

\subsection{Normalizing a~Solution}
\label{sec:def_normal}
In this section, we describe a~natural way of normalizing an~Eulerian solution~$D$. Informally, it can be viewed as follows.
Imagine that all arcs of~$D$ form one~circular thread, and that there is a~nail in every node~$s \in {\cal S}$ corresponding to an input string. We apply {\em ``gravitation''} to the thread, i.e., we replace every pair of arcs $(\pref(v), v)$, $(v, \suff(v))$ 
with a~pair $(\pref(v), \pref(\suff(v)))$, $(\pref(\suff(v)), \suff(v))$, if there is no nail in~$v$ and if this does not disconnect~$D$. We call this {\em collapsing}, see Figure~\ref{fig:collapsing}.

\begin{figure}[ht]
\begin{mypic}
\begin{scope}[minimum size=6mm]
%\draw[help lines] (0,0) grid (14,3);
\node[vertex] (a) at (0, 1.5) {$\pref(v)$};
\node[vertex,inner sep=1mm] (b) at (1.5, 2.5) {$v$};
\node[vertex] (c) at (3, 1.5) {$\suff(v)$};
\node[vertex] (d) at (1.5, 0.5) {$ \pref(\suff(v))$};
\draw[->,dashed] (a) -- (b);
\draw[->,dashed] (b) -- (c);
\draw[->] (a) -- (d);
\draw[->] (d) -- (c);

\begin{scope}[xshift=80mm]
\node[vertex] (a) at (0, 1.5) {\tt aba};
\node[vertex,inner sep=1mm] (b) at (1.5, 2.5) {\tt abac};
\node[vertex] (c) at (3, 1.5) {\tt bac};
\node[vertex] (d) at (1.5, 0.5) {\tt ba};
\draw[->,dashed] (a) -- (b);
\draw[->,dashed] (b) -- (c);
\draw[->] (a) -- (d);
\draw[->] (d) -- (c);
\end{scope}
\end{scope}
\end{mypic}
\caption{Collapsing a~pair of arcs is replacing a~pair of dashed arcs with a~pair of solid arcs: general case (left) and example (right). The~``physical meaning'' of this transformation is that to get {\tt bac} from {\tt aba} one needs to cut~{\tt a} from the beginning and append~{\tt c} to the end and these two operations commute.}
\label{fig:collapsing}
\end{figure}

A~formal pseudocode of the collapsing procedure is given in Algorithm~\ref{alg:collapse}. The pseudocode, in particular, reveals an important exception (not covered in Figure~\ref{fig:collapsing}):
if $|v|=1$, then $\pref(\suff(v))$ is undefined and we just remove the pair of arcs $(\pref(v), v)$ and $(v, \suff(v))$.


\begin{algorithm}[!ht]
\caption{Collapse}\label{alg:collapse}
\hspace*{\algorithmicindent} \textbf{Input:} hierarchical graph $HG(V,E)$, Eulerian solution~$D$, node~$v \in V$.
%\hspace*{\algorithmicindent} \textbf{Output:} a~superstring of~${\cal S}$ as a~path $D$~in the hierarchical graph.
\begin{algorithmic}[1]
\If{$(\pref(v), v), (v, \suff(v)) \in D$}
\State\label{alg:col} $D \gets D \setminus \{(\pref(v), v), (v, \suff(v))\}$
\If{$|v| > 1$}
\State $D \gets D \cup \{(\pref(v), \pref(\suff(v))), (\pref(\suff(v)), \suff(v))\}$
\EndIf
\EndIf
\end{algorithmic}
\end{algorithm}

Algorithm~\ref{alg:ca}, that we call Collapsing Algorithm (CA), uses the property described above to normalize any solution.
It drops down all pairs of arcs that are not needed for connectivity.
\ab{(Recall that a~set of edges is called an Eulerian solution if it is connected and goes through all initial nodes and $\varepsilon$.)}


%The normalizing 
%This way, the algorithm drops down all arcs of the doubled set~$D$ that are not needed for connectivity. See Algorithm~\ref{alg:ca} for a~formal pseudocode.

\begin{algorithm}[!ht]
\caption{Collapsing Algorithm (CA)}\label{alg:ca}
\hspace*{\algorithmicindent} \textbf{Input:} set of strings~$\mathcal{S}$, Eulerian solution~$D$ in~$HG$.\\
\hspace*{\algorithmicindent} \textbf{Output:} Eulerian solution $D'\colon |D'|\leq|D|$
% a~superstring of~${\cal S}$ as a~path $D$~in the hierarchical graph.
\begin{algorithmic}[1]
\For{level~$l$ in $HG$ in descending order}\label{alg:ca_for}
\For{all $v \in V$ s.t. $|v|=l$ in lexicographic order:}
\While{$(\pref(v), v), (v, \suff(v)) \in D$ and collapsing it keeps~$D$ an Eulerian solution}
\State $\text{\sc Collapse}(HG, D, v)$
\EndWhile
\EndFor
\EndFor
\State return $D$
\end{algorithmic}
\end{algorithm}

It is easy to show (we prove this formally in Claim~\ref{claim:zigzag} on page~\pageref{claim:zigzag}) that any normalized solution is of the form~\eqref{eq:zigzag}. But it is not true that every zig-zag solution of the form~\eqref{eq:zigzag} is a normalized solution: see Figure~\ref{fig:abnormalzigzag} for an example. The normalization procedure does not just turn a~solution into some standard form, but it may also decrease its length.

\begin{figure}[ht]
\begin{mypic}
\foreach \n/\x/\y in {ae/1/2, aa/2/2, ca/3/2}
  \node[inputvertex] (\n) at (\x,\y) {\tt \n};

\foreach \n/\x/\y in {a/2/1, e/1/1, c/3/1}
  \node[vertex] (\n) at (\x,\y) {\tt \n};

\node[vertex] (eps) at (2,0) {$\varepsilon$};
\node at (2,-1) {(a)};

\path (aa) edge[->,bend left] (a);
\path (a)    edge[->,bend left] (aa);
\path (eps) edge[->,bend left] (a);
\path (a)    edge[->,bend left] (eps);
\path (eps) edge[->,bend left=10] (a);
\path (a)    edge[->,bend left=10] (eps);

\foreach \f/\t in {a/ae, ae/e, e/eps, eps/c, c/ca, ca/a}
  \draw[->] (\f) -- (\t);
  
\begin{scope}[xshift=50mm]
\foreach \n/\x/\y in {ae/1/2, aa/2/2, ca/3/2}
  \node[inputvertex] (\n) at (\x,\y) {\tt \n};

\foreach \n/\x/\y in {a/2/1, e/1/1, c/3/1}
  \node[vertex] (\n) at (\x,\y) {\tt \n};

\node[vertex] (eps) at (2,0) {$\varepsilon$};
\node at (2,-1) {(b)};

\path (aa) edge[->,bend left] (a);
\path (a)    edge[->,bend left] (aa);

\foreach \f/\t in {a/ae, ae/e, e/eps, eps/c, c/ca, ca/a}
  \draw[->] (\f) -- (\t);
\end{scope}
\end{mypic}
\caption{(a)~An~Eulerian solution corresponding to the~permutation $({\tt ae}, {\tt aa}, {\tt ca})$. (b)~The solution from~(a) after normalization results in a shorter solution corresponding to the~permutation~$({\tt ca}, {\tt aa}, {\tt ae})$. \ab{This example also shows that although collapsing a~pair of edges is a~local change in the graph, it may drastically change the resulting superstring. In this case, it replaces a~superstring {\tt aeaaca} with a~shorter superstring {\tt caae}.}}
\label{fig:abnormalzigzag}
\end{figure}


 
