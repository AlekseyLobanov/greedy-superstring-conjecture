\clearpage
\section{Greedy Hierarchical Conjecture}
In this section, we present one more curious property of the Collapsing Algorithm. For this, we introduce the so called Greedy Hierarchical Algorithm (GHA).
%\subsection{Algorithm and Conjecture}
%In this section, we present a~greedy algorithm for
%finding a~superstring in the hierarchical graph. 
It constructs a~solution~$D$ in a~stingy way. 
Namely, the algorithm only adds arcs to~$D$ 
if it is absolutely necessary: either to balance the degree of a~node or to ensure connectivity 
(as $D$ must be Eulerian). 
More precisely, first it considers the input
strings~${\cal S}$. Since we assume that 
no $s \in {\cal S}$ is a~substring of another 
$t \in {\cal S}$, there is no down-path from~$t$ to~$s$ in $HG$. 
This means that any walk through $\varepsilon$ and ${\cal S}$ goes through the arcs $\{(\operatorname{pref}(s), s), (s, \operatorname{suff}(s)) \colon s \in {\cal S}\}$. The algorithm adds all of them to~$D$ and starts processing all the nodes level by level, from top to bottom. At each level, we process the nodes in the lexicographic order. If the degree of the current node~$v$ is imbalanced, we balance it by adding an appropriate number of incoming (i.e., $(\pref(v),v)$) or outgoing (i.e., $(v, \suff(v))$) arcs from the previous (i.e., lower) level. In the case where $v$ is balanced, we just skip it. The only exception when we cannot skip it is when {\em $v$~lies in an Eulerian component and $v$ is the last chance of this component to be connected to the rest of the arcs in~$D$}. We give an~example of such a~situation below. The pseudocode is given in~Algorithm~\ref{algo:gha}. Figure~\ref{fig:hgexa} shows a~few intermediate stages of the algorithm on our working sample dataset.


\begin{algorithm}[!ht]
\caption{Greedy Hierarchical Algorithm (GHA)}\label{algo:gha}
\hspace*{\algorithmicindent} \textbf{Input:} set of strings~${\cal S}$.\\
\hspace*{\algorithmicindent} \textbf{Output:} Eulerian solution~$D$.
\begin{algorithmic}[1]
\State $HG(V,E) \gets \text{hierarchical graph of ${\cal S}$}$ 
\State\label{alg:gha_init}$D \gets \{(\operatorname{pref}(s), s), (s, \operatorname{suff}(s)) \colon s \in {\cal S}\}$
\For{level $l$ from $\max\{|s| \colon s \in {\cal S}\}$ downto 1}\label{alg:for}
\For{node $v \in V$ with $|v|=l$ in the lexicographic order}
%\If{$\operatorname{upper-indegree}(v, D) \neq \operatorname{upper-outdegree}(v, D)$}
\If{$|\{(u,v) \in D \colon |u|=|v|+1\}| \neq |\{(v,w) \in D \colon |w|=|v|+1\}| $}
\State\label{alg:step6} balance the degree of $v$ in~$D$ by adding an appropriate number of lower arcs
\Else
\State\label{alg:else} ${\cal C} \gets \text{weakly connected component of $v$ in $D$}$
\State $u \gets \text{the lexicographically largest string among shortest strings in ${\cal C}$}$
\If{${\cal C}$ is Eulerian, $\varepsilon \not \in {\cal C}$, and $v = u$}
\State\label{alg:last} $D \gets D \cup \{(\pref(v), v), (v, \suff(v))\}$
\EndIf
\EndIf
\EndFor
\EndFor
\State return $D$
\end{algorithmic}
\end{algorithm}

\begin{figure}[!ht]
\begin{mypic}
\we{0}{0}{a}{
\foreach \f/\t/\a in {aa/aaa/10, aaa/aa/10, ca/cae/0, cae/ae/0, ae/aec/0, aec/ec/0, ee/eee/10, eee/ee/10}
  \path (\f) edge[hgedge,bend left=\a,draw=black,thick] (\t);
}

\we{57}{0}{b}{
\foreach \f/\t/\a in {aa/aaa/10, aaa/aa/10, ca/cae/0, cae/ae/0, ae/aec/0, aec/ec/0, ee/eee/10, eee/ee/10, aa/a/10, a/aa/10, c/ca/0, ec/c/0, ee/e/10, e/ee/10}
  \path (\f) edge[hgedge,bend left=\a,draw=black,thick] (\t);
}

\we{114}{0}{c}{
\foreach \f/\t/\a in {aa/aaa/10, aaa/aa/10, ca/cae/0, cae/ae/0, ae/aec/0, aec/ec/0, ee/eee/10, eee/ee/10, aa/a/10, a/aa/10, c/ca/0, ec/c/0, ee/e/10, e/ee/10, a/aa/10, aa/a/10, eps/c/10, c/eps/10, e/eps/10, eps/e/10, a/eps/10, eps/a/10}
\path (\f) edge[hgedge,bend left=\a,draw=black,thick] (\t);
}
\end{mypic}
\caption{(a)~After processing the $l=3$ level. (b)~After processing the $l=2$ level. Note that for the node {\tt aa} we add two lower arcs ($({\tt a}, {\tt aa})$ and $({\tt aa}, {\tt a})$) since otherwise the corresponding weakly connected component ($\{{\tt aa}, {\tt aaa}\}$) will not be connected to the rest of the solution. At the same time, when processing the node {\tt ae} we observe that it lies in a~weakly connected component that contains imbalanced nodes ({\tt ca} and {\tt ec}), hence there is no need to add two lower arcs to {\tt ae}. (c)~After processing the $l=1$ level. The resulting solution has length~10 and is, therefore, suboptimal (compare it with the optimal solution shown in Figure~\ref{fig:hgex}(c)).}
\label{fig:hgexa}
\end{figure}

One advantage of GHA over GA is that GHA is more flexible in the following sense. On every step, GA selects two strings and fixes tightly an order on them. GHA instead works to ensure connectivity. When the resulting set~$D$ is connected, an actual order of input strings is given by the corresponding Eulerian cycle through~$D$. This is best illustrated on a~toy example. For a~dataset $\mathcal{S}=\{{\tt ae}, {\tt ea}, {\tt ee}\}$, GA might produce a~suboptimal solution {\tt aeaee} if it merges the strings {\tt ae} and {\tt ea} at the first step. At the same time, it is not difficult to see that GHA finds an optimal solution for~$\mathcal{S}$. Another advantage of GHA is that, in contrast to GA, it solves {\em exactly} two well known polynomially solvable special cases of SCS: when the input strings have length at most two and when the input strings form a~$k$-spectrum of an unknown string. We prove this in Sections~\ref{sec:ghatwo} and~\ref{sec:ghaspectrum}. In Section~\ref{sec:ghatough}, we also show a~dataset where GHA produces a~solution that is almost two times longer than the optimal one.

We are now ready to state our second conjecture: the results of the Collapsing Algorithm and Greedy Hierarchical Algorithm coincide!
\newtheorem*{ghcc}{Greedy Hierarchical Conjecture}
\begin{ghcc}
For any set of strings~$\mathcal{S}$ and any Eulerian solution~$D$,
\[CA(\mathcal{S}, D \sqcup D) = GHA(\mathcal{S}) \, .\]
\end{ghcc}

\newtheorem*{wghcc}{Weak Greedy Hierarchical Conjecture}
\begin{wghcc}
GHA is a factor $2$ approximation algorithm for the Shortest Common Superstring problem.
\end{wghcc}

\section{Relations between the Conjectures}
\subsection{Equivalence of Collapsing and Greedy Hierarchical Conjectures}
\subsection{Greedy Implies Greedy Hierarchical}

\begin{theorem}
The Greedy Conjecture implies the Weak Greedy Hierarchical Conjecture: if the Greedy Algorithm is 2-approximate, then so is the Greedy Hierarchical Algorithm.
\end{theorem}
\begin{proof}
Consider a~permutation of the input strings. We say that it is a~{\em valid greedy permutation} if it can be constructed by the Greedy Algorithm: there is a~way of $n-1$ merges of $n$~input strings that leads to this permutation such that at every step the two merged strings have the largest overlap. To prove the theorem, we show that the permutation constructed by the Greedy Hierarchical Algorithm is a~valid greedy permutation.

Assume that $\pi=(s_1, \dots, s_n)$ is a~permutation constructed by GHA. Consider the following algorithm~$A$: it starts with a~sequence $(s_1, \dots, s_n)$ and on every step merges two neighbor strings in the sequence that have the largest overlap. It is a~greedy algorithm, but instead of considering all pairwise overlaps, it only considers overlap of neighbor strings in the sequence. Of course, in the end, this algorithm constructs exactly the permutation~$\pi$. To show that $\pi$ is a~valid greedy permutation, we show that at every iteration of~$A$ no two strings have longer overlap than the two strings merged by~$A$.

Assume, for the sake of contradiction, that at some iteration $A$~merges some two neighbor strings with overlap of length~$k$ whereas there are non-neighbor strings~$p$ and~$q$ with $v=\overlap(p,q)$, $|v|>k$. 
%Further assume that this is the earliest such iteration. 
At this point $p$~is a~merge of input strings $s_a, s_{a+1}, \dotsc, s_b$
and $q$~is a~merge of input strings $s_c, s_{c+1}, \dotsc, s_d$. 
Then, $v=\overlap(p,q)=\overlap(s_b,s_c)$. Since the algorithm~$A$
merges neighbor strings in order of decreasing overlap length, $p$~is merged with a~string~$q'$ such that $|\overlap(p,q')| \le k$ and $q$~is merged with a~string~$p'$ such that $|\overlap(p',q)| \le k$. This, in turn, means that the Eulerian solution around input strings~$s_b$ and~$s_c$ in GHA looks as shown in~Figure~\ref{fig:gagha}(a). Hence, $v$~is a~node with at least one incoming arc from the previous level and at least one outgoing arc to the previous level. Such a~pair of edges in the Eulerian solution constructed by GHA may only occur when~$v$ is the last chance of
its connected component to connect to the rest of the solution (see line~\ref{alg:last} of Algorithm~\ref{algo:gha}). This, in turn, means that right before the pair of arcs $(\pref(v), v)$ and $(v, \suff(v))$ were added to the Eulerian solution, there was an~Eulerian component where $v$~was the lexicographically largest among all shortest nodes. This component is shown schematically in~Figure~\ref{fig:gagha}(b). All overlap-nodes of this component lie on levels~$>k$. This means that all pairs of corresponding neighbor input strings are already merged at this stage. But then, $s_b$ and
$s_c$ already stay in the same merged string. This is a~contradiction with the assumption that the strings~$p$ and~$q$ are available for merging.

\begin{figure}
\label{fig:gagha}
\begin{mypic}
%\draw[help lines] (0,0) grid (14,6);
\node[inputvertex] (b) at (0,6) {$s_b$}; 
\node[inputvertex] (c) at (4,6) {$s_c$}; 
\node[vertex] (v) at (2,3) {$v$};
\node[vertex] (pv) at (1,2) {};
\node[vertex] (sv) at (3,2) {};
\draw[->] (pv) -- (v);
\draw[->] (v) -- (sv);
\draw[->,anypath] (b) -- (v);
\draw[->,anypath] (v) -- (c);

\node at (2,1) {(a)};

\begin{scope}[xshift=100mm]
\node[inputvertex] (b) at (0,6) {$s_b$}; 
\node[inputvertex] (c) at (4,6) {$s_c$}; 
\node[inputvertex] (d) at (2,6) {}; 
\node[inputvertex] (e) at (-2,6) {}; 
\node[vertex] (f) at (4,4) {};
\node[vertex] (g) at (0,3) {};
\node[vertex] (h) at (-0.5,5) {};
\node[vertex] (v) at (2,3) {$v$};
\node[vertex] (pv) at (1,2) {};
\node[vertex] (sv) at (3,2) {};
\draw[->] (pv) -- (v);
\draw[->] (v) -- (sv);

\foreach \f/\t in {b/v, v/c, c/f, f/d, d/g, g/e, e/h, h/b}
  \draw[->,anypath] (\f) -- (\t); 

\node at (2,1) {(b)};
\end{scope}
\end{mypic}
\caption{(a)~In the Eulerian solution the node $v=\overlap(s_b,s_c)$ has a~pair of lower arcs. (b)~For this reason, above~$v$, there is an~Eulerian component.}
\end{figure}
\end{proof}

\subsection{Cycle Covers}