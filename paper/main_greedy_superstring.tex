\documentclass[11pt,letterpaper]{article}
\usepackage[utf8]{inputenc}
\usepackage[english]{babel}
\usepackage{amsmath}
\usepackage{amsfonts}
\usepackage{amssymb}
\usepackage{amsthm}
\usepackage{fullpage}
\usepackage[colorinlistoftodos]{todonotes}
\usepackage[hidelinks]{hyperref}

\newtheorem{lemma}{Lemma}
\newtheorem{theorem}{Theorem}
\newtheorem{claim}{Claim}

\begin{document}
\sloppy
\author{}
\title{Collapsing Superstring Conjecture}
\maketitle

\section{Introduction}
The {\em shortest common superstring problem} (abbreviated as SCS) is:
given a~set of strings, find a~shortest string that contains all of them as
substrings. The currently record known approximation ratio is 
$2\frac{11}{23}$ due to Mucha~\cite{}.
The corresponding algorithm and its analysis are complicated.
At the same time, an~intriguing {\em greedy conjecture} remains open
for about 40 years already. It states that the following simple 
greedy algorithm is 2-approximate: while there are more than two strings 
in the set, take two of them with the maximum overlap and replace them
with their shortest superstring.

Most of the approaches for approximating SCS are based on an
{\em overlap graph}: it has input strings as nodes, any two nodes 
are joined by an~edge of weight equal to their overlap.  
While it is a~convenient graph structure, it does not seem to be sufficient
for proving the greedy conjecture.

In this paper, we continue the study of the so-called {\em hierarchical graph}
introduced by Golovnev et al.~\cite{}. This graph is designed specifically 
for the SCS problem and contains more useful information about input strings
that just all pairwise overlaps. We present simple and natural greedy algorithm
in the hierarchical graph. 
We demonstrate its usefulness by showing that it finds an optimal solution 
in two well-known polynomially solvable special cases: strings of length~2 and
a~$k$-spectrum of a~string.

We then conjecture that this greedy algorithm is 2-approximate. For this, we introduce an {\em even stronger} conjecture that we call 
{\em Collapsing Superstring Conjecture}. 
Roughly, it says that it is possible to transform a~doubled optimal 
solution into a~greedy solution. 
The corresponding transformation, that we call {\em collapsing}, 
is just replacing two edges $a\alpha \to a\alpha b \to \alpha b$ 
by two edges $a\alpha \to \alpha \to \alpha b$. 
We report on computational experiments that verified the 
conjecture on many datasets (both hand-crafted and generated randomly
according to various distributions). 
We then support the Collapsing Superstring Conjecture by 
proving that it holds for a~special case when the input strings have length~3.

The Collapsing Superstring Conjecture implies immediately that the Greedy Hierarchical Algorithm is 2-approximate. Surprisingly, it seems to be much stronger in the following sense. Let $GS$ be the set of edges of a~greedy solution and let $DOS$ be the set of edges of a~collapsed double optimal solution. For proving 2-approximability, it suffices to show that $|GS| \le |DOS|$. One way of showing this is to prove that $GS \subseteq DOS$. The conjecture, at the same time, states that this inclusion holds with equality.

\end{document}