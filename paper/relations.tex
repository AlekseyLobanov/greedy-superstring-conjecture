\section{Relations between the Conjectures}
In this section we prove some relations between the Collapsing and Greedy conjectures. Namely, in Section~\ref{sec:equiv} we prove the equivalence of Collapsing and Greedy Hierarchical conjectures. In Section~\ref{sec:gr_im_wghc} we prove that the standard Greedy Conjecture implies Weak Hierarchical Greedy Conjecutre (which is sufficient for a simple 2-approximate greedy algorithm for SCS). Finally, it is easy to see that Greedy Hierarchical Conjecture implies its weak version: indeed, if every doubled solution results in the solution obtained by GHA, then GHA does not exceed twice the optimal superstring length. In Figure~\ref{fig:relations} we show the proven relations between the conjectures, together with 2-approximate algorithm which follow from each of the conjecture.
%\begin{figure}
%\begin{mypic}
%\tikzstyle{r}=[rectangle,inner sep=2mm,draw,text centered]
%
%\node[r] (w) at (5,3) {Weak Greedy Hierarchical Conjecture};
%\node[r] (cc) at (0,5) {Collapsing Conjecture}; 
%\node[r] (gc) at (10,5) {Greedy Conjecture}; 
%
%\node[r] (ca) at (0,1) {CA is 2-approximate};
%\node[r] (gha) at (5,1) {GHA is 2-approximate};
%\node[r] (ga) at (10,1) {GA is 2-approximate};
%
%%\draw[help lines] (0,0) grid (14,6);
%
%\foreach \f/\t in {cc/ca, cc/w, gc/ga, gc/w, w/gha, ga/gha}
%  \draw[->] (\f) -- (\t);
%\end{mypic}
%\caption{Collapsing and Greedy Hierarchical Conjectures are equivalent. They imply the weak version of the Greedy Hierarchical Conjecture, which also follows from the standard Greedy Conjecture. Each conjecture implies that the corresponding algorithm finds a $2$-approximate solution for SCS.}
%\label{fig:relations}
%\end{figure}
\begin{figure}
\begin{mypic}
\tikzstyle{r}=[rectangle,inner sep=2mm,draw,text centered, text width=40mm]

\node[r] (w) at (5.5,3) {Weak Greedy Hierarchical Conjecture};
\node[r] (cc) at (0,5) {Collapsing Conjecture}; 
\node[r] (gc) at (0,1) {Greedy Conjecture};
\node[r] (ghc) at (0,3) {Greedy Hierarchical Conjecture}; 

\node[r] (ca) at (12,5) {CA is 2-approximate};
\node[r] (gha) at (12,3) {GHA is 2-approximate};
\node[r] (ga) at (12,1) {GA is 2-approximate};

%\draw[help lines] (0,0) grid (14,6);

\foreach \f/\t in {cc/ghc, cc.east/w, ghc/w, gc.east/w, ghc/cc}
  \draw[->] (\f) -- (\t);
  
\foreach \f/\t in {cc/ca, w/gha, gc/ga}
  \draw[dashed,->] (\f) -- (\t);
 

%\foreach \f/\t in {cc/ca, cc/w, gc/ga, gc/w, w/gha, ga/gha}
 % \draw[->] (\f) -- (\t);
\end{mypic}
\caption{Relations between the conjectures (left), and the 2-approximate algorithms they imply (right). 
Collapsing and Greedy Hierarchical Conjectures are equivalent. They imply the weak version of the Greedy Hierarchical Conjecture, which also follows from the standard Greedy Conjecture. Each conjecture implies that the corresponding algorithm finds a $2$-approximate solution for SCS.}
\label{fig:relations}
\end{figure}

\subsection{Equivalence of Collapsing and Greedy Hierarchical Conjectures}
\label{sec:equiv}
In this section we prove the equivalence of Collapsing Conjecture and Greedy Hierarchical Conjecture. Recall that Collapsing Conjecture claims that for any pair of Eulerian solutions $D_1$ and $D_2$ for the input strings ${\cal S}$, we have 
\[CA({\cal S}, D_1 \sqcup D_1) =  CA({\cal S}, D_2 \sqcup D_2)\, .\]
 The Greedy Hierarchical Solution extends this statement to 
 \[CA({\cal S}, D_1 \sqcup D_1) =  GHA(\mathcal{S}) \, .\] Greedy Hierarchical Conjecture trivially implies Collapsing conjecture, and in order to prove their equivalence, it suffices to show that the collapsing procedure applied to the doubled GHA solution results in the GHA solution: 
 \[CA({\cal S}, GHA(\mathcal{S}) \sqcup GHA(\mathcal{S})) =  GHA(\mathcal{S}) \, .\]
 
 \begin{theorem}
For any set of strings $\mathcal{S}$,
\[
			CA(\mathcal{S}, GHA(\mathcal{S}) \sqcup GHA(\mathcal{S})) = GHA(\mathcal{S}) \, .
\]
\end{theorem}
\begin{proof}
Let us denote two copies of the $GHA(\mathcal{S})$ solution by $B$ and $R$, which sttand for a blue-copy and a red-copy. We will prove the theorem statement by showing that $CA(R \sqcup B)$ collapses all arcs of $B$ and keeps untouched the arcs of $R$, as this implies that $CA(R \sqcup B)=R=GHA(\mathcal{S})$. For this, without loss of generality assume that the Collapsing Algorithm collapses blue arcs first, that is, if for a vertex $v$ $CA$ can collapse a blue pair of arcs $(\pref(v), v), (v, \suff(v))$, it does so. Recall that $CA$ proceeds vertices in the descending order of levels (Algorithm~\ref{alg:ca}, line~\ref{alg:ca_for}). We will prove that before processing level $l$, all the arcs above it (that is arcs with at least one vertex at a level > $l$) do satisfy the desired property: all blue arcs are collapsed, and all red arcs are untouched.

The base case trivially holds for $l := \max\{|s| \, : \, s\in\mathcal{S}\}$, since the set of arcs above the level $l$ is empty. Assume the claim is true for the level $k > 0$, and let us prove the claim for the level $k - 1$. Note that regardless of the number of collapse operations applied to $B$, $B$ remains a set of walks: indeed, the collapse procedure keeps the balance of incoming and outgoing arcs for each vertex. Therefore, if for a vertex $v$ at level $k$ there is an arc $(\pref(v), v)$ in $B$, then there is also an arc $(v, \suff(v))$ in $B$, and vice versa. Recall that $CA$ collapses blue arcs when possible, and since every vertex has the same number of blue incoming and outgoing arcs, all pairs collapsed at the level $k$ are monotone. 
	
	Now let us show, that no red pair can be collapsed. Indeed, if for some vertex $v$ at level $k$ there is a red pair $(\pref(v), v)$, $(v, \suff(v))$, then by construction of $GHA$ $v$ is either in $\mathcal{S}$ or is the last chance of the corresponding component $\mathcal{C} \ni v$ to be connected to the remaining arcs in $R$ (note, that the first case is subcase of the second one, as then $\mathcal{C}$ contains only one vertex). It follows, that if $CA$ does collapse such a pair, then $v$ has no blue arcs (as they have been collapsed before the red arcs), and all other vertices in the component $\mathcal{C}$ at the level $k$ collapsed all arcs (since $v$ is the last vertex in $\mathcal{C}$ in lexicographic order). Therefore, this pair is also the last chance of $\mathcal{C}$ to be connected to the rest arcs in $R$, thus, $CA$ cannot collapse it.
	
	It remains to show that all blue pairs at level $k$ are collapsed. This trivially holds because no red pair is collapsed, and, thus, the connectivity of $R \sqcup B$ is maintained by $R$. This finishes the proof.
\end{proof}


\subsection{Greedy Implies Greedy Hierarchical}
\label{sec:gr_im_wghc}
Consider a~permutation of the input strings. We say that it is a~{\em valid greedy permutation} if it can be constructed by the Greedy Algorithm: there exist $n-1$ merges of the $n$~input strings that lead to this permutation such that at every step the two merged strings have the largest overlap. We will prove that GHA always returns a solution which corresponds to a greedy permutation of the input string. That is, while the standard Greedy Algorithm does not determine how to break ties, the Greedy Hierarchical Algorithm is a specific instantiation of the Greedy Algorithm with some tie-breaking rule.

We will use the following simple property of solutions constructed by the GHA algorithm.
\begin{claim}
\label{claim:zigzag}
Let $D$ be an Eulerian solution constrtucted by GHA. Then $D$ has a ``zig-zag'' form as in~\eqref{eq:zigzag}. 
\end{claim}
\begin{proof}
First we prove that $D$ is normalized, that is, any application of the collapsing procedure of Algorithm~\ref{alg:collapse} to $D$ will violate the property of Eulerian solution. Indeed, Algorithm~\ref{alg:collapse} can only collapse pairs of arcs of the form $(\pref(s), s), (s, \suff(s))$. The Greedy Hierarchical Algorithm adds such pairs to its solution in two cases: (i) $s$ is an input string (line~\ref{alg:gha_init} of Algorithm~\ref{algo:gha}); (ii) $s$ is the the lexicographically largest among the shortest strings in its Eulerian component (line~\ref{alg:last} of Algorithm~\ref{algo:gha}). Now note that in the former case, the collapsing procedure applied to $s$ would violate the property that $D$ must contain all input strings, and in the latter case, the collapsing procedure would violate the connectivity property of $D$.

We finish the proof by showing that every normalized solution is of the form~\eqref{eq:zigzag}. Let $\pi=(s_1, \dots, s_n)$ be  the permutation of the input strings corresponding to a normalized Eulerian solution $D$. Let us follow the arcs of $D$ in the order of the permutation $\pi$, and let $P$ be the set of arcs between the input strings $s_i$ and $s_{i+1}$. We will prove that $P$ is the union of the sets of arcs of the paths $s_i \to \overlap(s_i,s_{i+1})$ and $\overlap(s_i,s_{i+1})\to s_{i+1}$. If $P$ conatins a pair of consecutive up- and down-arcs, that is, there exists a pair of arcs $(\pref(s), s), (s, \suff(s))$ in $P$, then this pair would have been collapsed by Algorithm~\ref{alg:collapse}, line~\ref{alg:col}. Therefore, the path $P$ consists of a number of down-arcs followed by a number of up-arcs. Itt remains to show that the number of down-arcs in $P$ is $d=|s_i|-|\overlap(s_i,s_{i+1})|$. Note that be the definition of $\overlap(\cdot,\cdot)$, the number of down arcs is at least $d$. On the other hand, if the number of down-arcs in $P$ is strictly greater than $d$, then both the down-path and up-path in $P$ contain the vertexd $\overlap(s_i, s_{i+1})$. This implies that the pair of arcs  $(\pref(s), s), (s, \suff(s))$ for $s=\overlap(s_i, s_{i+1})$ would have been collapsed by Algorithm~\ref{alg:collapse}, line~\ref{alg:col}, as it does not violate the connectivity of the solution. Therefore, the number of down-arcs in $P$ is exactly $d$, which implies that $P$ is the path $s_i \to \overlap(s_i,s_{i+1})$ followed by the path $\overlap(s_i,s_{i+1})\to s_{i+1}$.
\end{proof}

\begin{theorem}
\label{thm:gr_im_wghc}
Every permutation $\pi=(s_1, \dots, s_n)$ of the input strings constructed by GHA is a greedy permutation.
\end{theorem}
\begin{proof}
Consider the following algorithm~$A$: it starts with the sequence $(s_1, \dots, s_n)$ obtained by GHA, and at every step it merges two neighboring strings in this sequence that have the largest overlap. It is a~greedy algorithm, but instead of considering all pairwise overlaps, it only considers overlaps of neighboring strings in the sequence. Of course, in the end, this algorithm constructs exactly the permutation~$\pi$. To show that $\pi$ is a~valid greedy permutation, we show that at every iteration of~$A$ no two strings have longer overlap than the two strings merged by~$A$.

Consider, for the sake of contradiction, the first iteration when the algorithm $A$~merges some pair of neighboring strings with overlap of length~$k$ whereas there are non-neighboring strings~$p$ and~$q$ with $v=\overlap(p,q)$, $|v|>k$. 
%Further assume that this is the earliest such iteration. 
At this point, $p$~is a~merger of input strings $s_a, s_{a+1}, \dotsc, s_b$
and $q$~is a~merger of input strings $s_c, s_{c+1}, \dotsc, s_d$. 
Then, from the assumption that no input string contains another input string, we have that $v=\overlap(p,q)=\overlap(s_b,s_c)$. Since the algorithm~$A$
merges neighboring strings in the decreasing order of overlap lengths, we have that $|\overlap(s_b,s_{b+1})| \le k <|v|$ and $|\overlap(s_{c-1},c_c)| \le k < |v|$.\footnote{In the case when $s_b$ is the last string in the solution (or $s_c$ is the first string in the solution) we think of it being followed by $\varepsilon$, and $|\overlap(s_b,\varepsilon)|=0<|v|$ still holds.} 

Now we consider the Eulerian solution $D$ constructed by GHA in the hierarchical graph.
By Claim~\ref{claim:zigzag}, $D$ has a ``zig-zag'' form, thus, it contains all arcs from the path $s_b\to\overlap(s_{b}, s_{b+1}) \to s_{b+1}$, and all arcs from the path $s_{c-1}\to\overlap(s_{c-1}, s_{c}) \to s_{c}$. Recall that $v=\overlap(s_b,s_c)$, and that $|\overlap(s_b,s_{b+1})| <|v|$ and $|\overlap(s_{c-1},c_c)|< |v|$. In particular, the paths $s_b\to\overlap(s_{b}, s_{b+1})$ and $\overlap(s_{c-1}, s_{c}) \to s_{c}$ pass throught the vertex $v$, which implies that the vertex $v$ in the solution $D$ has at least one incoming arc from the previous level and at least one outgoing arc to the previous level (see~Figure~\ref{fig:gagha}(a)). Such a~pair of arcs in the Eulerian solution $D$ constructed by GHA may only occur when~$v$ is the last chance of
its connected component to be connected to the rest of the solution (see line~\ref{alg:last} of Algorithm~\ref{algo:gha}). This, in turn, implies that right before the pair of arcs $(\pref(v), v)$ and $(v, \suff(v))$ were added to the Eulerian solution, there was an~Eulerian component where $v$~was the lexicographically largest among all shortest nodes. This component is shown schematically in~Figure~\ref{fig:gagha}(b). All overlap-nodes (the nodes which are equal to $\overlap(s_i,s_{i+1})$) of this component lie on levels~$\geq k$. Therefore, all pairs of the corresponding neighboring input strings are already merged at this stage. But then, $s_b$ and
$s_c$ already belong to the same merged string.
It will be proven later (in Lemma~\ref{lem:edgecases}) that the resulting solution will include only one copy of arcs $(\pref(v), v)$ and $(v, \suff(v))$. Summing up, the vertex $v$ can be firstly traversed by any Eulerian path only using the arc $(\pref(v), v)$, and after this the overlying component containing $s_b$ and $s_c$ will be fully traversed, otherwise the graph will be disconnected.
In turn, this means that after considering all overlaps of length $|v|$ $s_b$ and $s_c$ will be already merged into some string, so they can't be merged once more.
This contradicts the assumption that the strings~$p$ and~$q$ can be merged.

\begin{figure}
\begin{mypic}
%\draw[help lines] (0,0) grid (14,6);
\node[inputvertex] (b) at (0,6) {$s_b$}; 
\node[inputvertex] (c) at (4,6) {$s_c$}; 
\node[vertex] (v) at (2,3) {$v$};
\node[vertex] (pv) at (1,2) {};
\node[vertex] (sv) at (3,2) {};
\draw[->] (pv) -- (v);
\draw[->] (v) -- (sv);
\draw[->,anypath] (b) -- (v);
\draw[->,anypath] (v) -- (c);

\node at (2,1) {(a)};

\begin{scope}[xshift=100mm]
\node[inputvertex] (b) at (0,6) {$s_b$}; 
\node[inputvertex] (c) at (4,6) {$s_c$}; 
\node[inputvertex] (d) at (2,6) {}; 
\node[inputvertex] (e) at (-2,6) {}; 
\node[vertex] (f) at (4,4) {};
\node[vertex] (g) at (0,3) {};
\node[vertex] (h) at (-0.5,5) {};
\node[vertex] (v) at (2,3) {$v$};
\node[vertex] (pv) at (1,2) {};
\node[vertex] (sv) at (3,2) {};
\draw[->] (pv) -- (v);
\draw[->] (v) -- (sv);

\foreach \f/\t in {b/v, v/c, c/f, f/d, d/g, g/e, e/h, h/b}
  \draw[->,anypath] (\f) -- (\t); 

\node at (2,1) {(b)};
\end{scope}
\end{mypic}
\caption{(a)~In the Eulerian solution the node $v=\overlap(s_b,s_c)$ has a~pair of lower arcs. (b)~For this reason, above~$v$, there is an~Eulerian component.}
\label{fig:gagha}
\end{figure}
\end{proof}

Theorem~\ref{thm:gr_im_wghc} has two immediate corollaries.
\begin{corollary}
The Greedy Conjecture implies the Weak Greedy Hierarchical Conjecture: if the Greedy Algorithm is 2-approximate, then so is the Greedy Hierarchical Algorithm.
\end{corollary}
Since every valid greedy permutation is a $3.5$-approximation to the Shortest Common Superstring problem~\cite{KS2005}, we have the following corollary.
\begin{corollary}
GHA is a factor $3.5$ approximation algorithm for the Shortest Common Superstring problem.
\end{corollary}