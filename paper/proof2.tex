\subsection{Case Analysis}
\label{sec:proof2}
We denote the level of the vertex $ s $ as $ \lvl (s) $, and consider the three possible values of $\lvl(s)$.


\paragraph{Analysis of vertices of the level \texorpdfstring{$3$}{3}}

In this case $ s \in {\cal S} $, and vertex $ s $, is incident only to arcs $ (\pref (s) \to s)$ and $(s \to \suff (s)) $. The Collapse procedure leaves only one instance of each such arc. Hence, in the graph $ G'_{cl} $, $ C'_{cl} (s) = \{s \} $. This implies that (\ref{eqn:contradiction}) does not hold in this case.

\paragraph{Analysis of vertices of the level \texorpdfstring{$2$}{2}}

Assume $ C'_{cl} (s) \backslash V (s) \ne\emptyset$. Choose a vertex $ t\in C'_{cl} (s) \backslash V (s)$ with the shortest weak path to $s$. Let $P = v_1 - v_2 - \ldots - v_n $ be a weak path from $s$ to $t$. In particular, $s=v_1$ and $t=v_n$.

Note that if $ \lvl (u) = 1 $, then $ u \notin V (s) $. Then $ \lvl (v_i) \ge 2 $ holds for $ i<n $, since otherwise there would be a vertex with a shorter weak path to $s$. Since for every arc $(a,b)$ of the hierarchic graph $ | \lvl (a) - \lvl (b) | = 1 $,  the levels of vertices (possibly until the last one) of the path $P$ alternateas follows: $ \lvl (v_1) = 2, \lvl (v_2) = 3, \lvl (v_3) = 2,\ldots$. See an example of $P$ in Figure~\ref{fig:lvl2path6}.

\begin{figure}[ht]
\begin{center}

\begin{tikzpicture}[xscale=1.3] 
  \begin{scope}
    \foreach \x/\y/\n in {2/3/v2, 4/3/v4}
      \node[inputvertex] (\n) at (\x,\y) {\n};
      
    \foreach \x/\y/\n in {1/2/v1, 3/2/v3, 5/2/v5, 6/1/v6}
      \node[vertex] (\n) at (\x,\y) {\n};

    
    \node[vertex] (e) at (3,0) {$\varepsilon$};
    
    \node[draw=none] at (0,0) {};
    
    \foreach \y in {0,...,3}
      \node[draw=none] at (-1,\y) {\y};
      
    \path (v1) edge[->] (v2);
    \path (v2) edge[->] (v3);
    \path (v3) edge[->] (v4);
    \path (v4) edge[->] (v5);
    \path (v5) edge[->] (v6);
  \end{scope}
\end{tikzpicture}
\end{center}

\caption{An example of a path $P$ for $n=6$.}\label{fig:lvl2path6}
\end{figure}

For a vertex $v_{2k}$ at the $\lvl (v_{2k}) = 3 $, let us denote the by $u=\pref(\suff(v_{2k}))$. The meaning of this vertex is that the pair of arcs $( v_{2k-1}, v_{2k}), (v_{2k}, v_{2k + 1})$ can potentially be collapsed into $( v_{2k-1}, u_{2k}), (u_{2k}, v_{2k + 1})$. Recall that we are considering Case~4 which means $a=b=1$, and $s=v_1$ has an up-edge and a down-edge. That is, $ G_{cl} $ contains arcs $ (u_0 \to v_1) $ and $ (v_1 \to u_2) $, where $ u_0 = \pref (s) $ and $ u_2 = \suff (s) $. 

We also remark that if there is a pair of arcs $ (u_{2k}, v_{2k + 1}) $ and $ (v_{2k + 1}, u_{2k + 2}) $ in the graph, then there is a vertex $ u_{2k + 1} $, into which these arcs could be collapsed.

Since the arcs $ (v_{2k-1}, v_{2k}) $ and $ (v_{2k} , v_{2k + 1}) $ are the only ones that are incident to the vertex $ v_{2k} \in S $ in the hierarchical graph, the initial (doubled) solution had two copies of each of them. Moreover, since $ \lvl (s) = 2 $, all the vertices of the level $ 3 $ have already been examined, and therefore at least one collapse has been performed from each vertex $ v_ {2k} $ at the level $ 3 $. Thus, at some point the graph contained the arcs $ (v_{2k-1} , u_{2k}) $ and $ (u_{2k}, v_{2k + 1}) $ (see Figure~\ref {fig:lvl2pathuv} for an example).

\begin{figure}[ht]
\begin{center}

\begin{tikzpicture}[xscale=1.3] 
  \begin{scope}
  \foreach \x/\y/\n in {2/3/v2, 4/3/v4}
      \node[inputvertex] (\n) at (\x,\y) { \n};
    \foreach \x/\y/\n in {1/2/v1, 3/2/v3, 5/2/v5, 6/1/v6}
      \node[vertex] (\n) at (\x,\y) { \n};
    \foreach \x/\y/\n in {0/1/u0, 1/0/u1, 2/1/u2, 3/0/u3, 4/1/u4, 5/0/u5}
      \node[vertex] (\n) at (\x,\y) { \n};
    
    \foreach \y in {0,...,3}
      \node[draw=none] at (-1,\y) {\y};
      
    \path (v1) edge[->] (v2);
    \path (v2) edge[->] (v3);
    \path (v3) edge[->] (v4);
    \path (v4) edge[->] (v5);
    \path (v5) edge[->] (v6);
    
    \draw[->] (u0) to node [sloped] {\(\not\quad\)} (v1);
    \draw[->] (v1) to node [sloped] {\(\not\quad\)} (u2);
    \path (u2) edge[->, dashed] (v3);
    \path (v3) edge[->, dashed] (u4);
    \path (u4) edge[->, dashed] (v5);
  \end{scope}
\end{tikzpicture}
\end{center}

\caption{An example of arrangement of the vertices $ v_i $ and $ u_i $ for $ n = 6 $. Note that $u_1=u_3=u_5=\varepsilon$. The graph contained the dashed arcs at some point. Removal of arcs $(u_0, v_1), (v_1, u_2)$ would violate weak connectivity of the graph.}\label{fig:lvl2pathuv}
\end{figure}

Now we will use the fact that the pair of edges $ (u_0, v_1) $ and $ (v_1, u_2) $ has not been collapsed. This means that after this collapse the graph would loose weak connectivity. Note that $ u_1 = \varepsilon $; after the collapse $ u_0 $ and $ u_2 $ would be weakly connected through $ \varepsilon $. This means that the weak connectivity could only be violated because there is no other weak path between $ v_1 $ and $ u_2 $.

\begin{claim}
\label{lem:pathcollapse}
For all $ k $ such that $ 3 \le 2k + 1 \le n $ at least one collapse was performed from the vertex $ v_{2k + 1} $.
\end{claim}
\begin{proof}
Suppose that at least one collapse is made from the vertex $ v_{2k '+ 1} $ for all $ k' $ such that $ 3 \le 2k '+ 1 <2k + 1 $, and the vertex $ v_{2k + 1} $ was not collapsed. This means that the arc $ (u_{2k}, v_{2k + 1}) $belongs to the graph at the current iteration.

By definition, after a collapse from the vertex $ v_{2k '+ 1} $, a pair of arcs $ (u_{2k'}, u_{2k '+ 1}), (u_{2k' + 1}, u_{2k '+ 2}) $ is added to the graph. Moreover, any of these arcs can disappear from the graph only after a collapse from the vertex $ u_ {2k '+ 1} $. But $\lvl(u_ {2k '+ 1})= 1 $, and therefore it has not yet been visited by the Collapsing Algorithm. Therefore, both edges belong to the graph at the current iteration.

Then there exists a weak path $ Q $ in the graph of the following form:

$$
Q = v_1 - v_2 - \ldots - v_{2k+1} - u_{2k} - u_{2k-1} - \ldots u_2.
$$

It means that $v_1$ and $u_2$ are weakly connected, which leads to a contradiction.
\end{proof}

From Claim~\ref{lem:pathcollapse} we know that each $v_{2k+1}$ was collapsed. This implies that at the current iteration, for every $k$ such that $3\le 2k+1\le n$, the graph contains the following weak path $ u_2 - \ldots - u_{2k} - u_{2k + 1} $. We will now show that $v_1$ and $u_2$ are weakly connected. To this end, we consider two cases $ \lvl (v_{n}) = 2 $ and $ \lvl (v_{n}) = 1 $. 

If $ \lvl (v_{n}) = 1 $, then, since every vertex $ v_{2k + 1} $ was collapsed at least once, there exist arcs $(u_{n-2},u_{n-1} ) $ and $ (u_{n-1}, v_n) $ in the graph. Hence, there exists a weak path of the form $ v_1 - v_2 - \ldots - v_n - u_{2k + 1} - \ldots - u_3 - u_2 $, and the vertices $ v_1 $ and $ u_2 $ are weakly connected (see an example in Figure~\ref{fig:lvl2n6pathvu}).

If $ \lvl (v_{n}) = 2 $, then $ n = 2k + 1 $ for some $ k $. Hence, by the Claim~\ref{lem:pathcollapse}, at least one collapse was made from the vertex $ v_{2k + 1} $. On the other hand, the vertex $ v_{2k + 1} \notin V (s) $ (since we chose $t=v_n=v_{2k+1}$ as a vertex not from $V(s)$). Since $v_{2k+1}$ has not been examined yet, it could not be collapsed yet (see an example in Figure~\ref{fig:lvl2n5pathvu}).

\begin{figure}[ht]
\begin{center}

\begin{tikzpicture}[xscale=1.3] 
  \begin{scope}
    \foreach \x/\y/\n in {2/3/v2, 4/3/v4}
      \node[inputvertex] (\n) at (\x,\y) { \n};
    \foreach \x/\y/\n in {1/2/v1, 3/2/v3, 5/2/v5, 6/1/v6}
      \node[vertex] (\n) at (\x,\y) { \n};
    \foreach \x/\y/\n in {0/1/u0, 1/0/u1, 2/1/u2, 3/0/u3, 4/1/u4, 5/0/u5}
      \node[vertex] (\n) at (\x,\y)  { \n};
    
    \foreach \y in {0,...,3}
      \node[draw=none] at (-1,\y) {\y};
      
    \path (v1) edge[->, ultra thick] (v2);
    \path (v2) edge[->, ultra thick] (v3);
    \path (v3) edge[->, ultra thick] (v4);
    \path (v4) edge[->, ultra thick] (v5);
    \path (v5) edge[->, ultra thick] (v6);
    
    \draw[->] (u0) to node [sloped] {\(\not\quad\)} (v1);
    \draw[->] (v1) to node [sloped] {\(\not\quad\)} (u2);
    \path (u2) edge[->, ultra thick] (u3);
    \path (u3) edge[->, ultra thick] (u4);
    \path (u4) edge[->, ultra thick] (u5);
    \path (u5) edge[->, ultra thick] (v6);
  \end{scope}
\end{tikzpicture}
\end{center}

\caption{A weak path connecting $v_1$ and $u_2$ for $n=6$.}\label{fig:lvl2n6pathvu}
\end{figure}

\begin{figure}[ht]
\begin{center}

\begin{tikzpicture}[xscale=1.3] 
  \begin{scope}
    \foreach \x/\y/\n in {2/3/v2, 4/3/v4}
      \node[inputvertex] (\n) at (\x,\y) { \n};
    \foreach \x/\y/\n in {1/2/v1, 3/2/v3, 5/2/v5}
      \node[vertex] (\n) at (\x,\y) { \n};
    \foreach \x/\y/\n in {0/1/u0, 1/0/u1, 2/1/u2, 3/0/u3, 4/1/u4}
      \node[vertex] (\n) at (\x,\y) { \n};
    
    \foreach \y in {0,...,3}
      \node[draw=none] at (-1,\y) {\y};
      
    \path (v1) edge[->, ultra thick] (v2);
    \path (v2) edge[->, ultra thick] (v3);
    \path (v3) edge[->, ultra thick] (v4);
    \path (v4) edge[->, ultra thick] (v5);
    
    \draw[->] (u0) to node [sloped] {\(\not\quad\)} (v1);
    \draw[->] (v1) to node [sloped] {\(\not\quad\)} (u2);
    \path (u2) edge[->, ultra thick] (u3);
    \path (u3) edge[->, ultra thick] (u4);
    \path (u4) edge[->, ultra thick] (v5);
  \end{scope}
\end{tikzpicture}
\end{center}

\caption{A weak path connecting $v_1$ and $u_2$ for $n=5$.}\label{fig:lvl2n5pathvu}
\end{figure}

\paragraph{Analysis of vertices of the level \texorpdfstring{$1$}{1}}

In this case, all the vertices at the levels $ 2 $ and $ 3 $ are already considered.

Note that the collapse from the vertex $ s $ can potentially violate only the connectivity of $ s $ and $ \varepsilon $. Therefore $ \varepsilon \notin C'_{cl} (s) $. Let's take an arbitrary vertex $ t $ from the statement \ref{eqn:contradiction}. From what has just been said, it follows that $ \lvl (t) = 1 $.

In view of the fact that $ t $ is connected with $ s $, the vertex $ t $ is incident to at least one edge. But it can not be connected with the vertex on the level $ 0 $, because only $ \varepsilon $ is located at this level, and $ \varepsilon \notin C'_{cl} (s) $. Hence, $ t $ is incident to at least one edge leading to the level $ 2 $. Moreover, since $ \bal (t) = 0 $, at least one such edge comes to $ t $, and at least one such edge leaves $ t $.

Starting from this point, we will call all vertices as strings which they correspond to. So, $ t $ corresponds to some string $ \tt b $ consisting of one symbol. We have just shown that there are edges $ ({\tt xb} \to {\tt b}) $ and $ ({\tt b} \to {\tt bc}) $ for some symbols $ {\tt x} $ and $ {\tt c} $.

First we prove several auxiliary lemmas.

\begin{lemma}
\label{lem:nobeps}
In the graph there neither was an edge $ ({\tt b} \to \varepsilon) $ nor an edge $ (\varepsilon \to {\tt b}) $.
\end{lemma}
\begin{proof}
Let at least one of such edges appear in the graph at some iteration. According to the algorithm, such an edge can disappear from the graph only after the collapse in the vertex $ {\tt b} $. Since this vertex has not yet been examined, such an edge is present in the graph at this moment. But this means that $ \varepsilon \in C'_{cl} (s) $. Contradiction.
\end{proof}

\begin{lemma}
\label{lem:bccsave}
Each edge of the form \sedge{bc}{c}, \sedge{b}{bc}, or \sedge{ab}{b} that ever appeared in the graph was not collapsed until the current iteration.
\end{lemma}
\begin{proof}
Each edge of the form \sedge{bc}{c} can only be collapsed with the edge \sedge{b}{bc}. Then, if a collapse actually occurs from the vertex $ {\tt bc} $, it leads to the appearance of the edge $ ({\tt b} \to \varepsilon) $. But according to the lemma \ref{lem:nobeps}, there never were any such edges in the graph. Contradiction.

The proofs for \sedge{b}{bc} and \sedge{ab}{b} are the same.
\end{proof}

\begin{lemma}
\label{lem:bbcone}
At the current iteration, there is exactly one instance of the edge \sedge{b}{bc}.
\end{lemma}
\begin{proof}
Let there be at least two such edges. Then at least two different edges must come from $ {\tt bc} $.

Suppose that there exists at least one edge \sedge{bc}{c}. But then, if we make a collapse from the vertex $ {\tt bc} $, the vertices $ {\tt b} $, $ {\tt c} $ and $ {\tt bc} $ still remain weakly connected by edges \sedge{b}{bc}, $ ({\tt b} \to \varepsilon) $ and $ (\varepsilon \to {\tt c}) $. Hence, this collapse is correct (see figure \ref{fig:lvl1lemmabcc}). But then the collapsing algorithm during examination of the vertex $ {\tt bc} $, should have made at least one more collapse. It contradicts the fact that it does the maximum possible number of correct collapses.

\begin{figure}[ht]
\begin{center}

\begin{tikzpicture}[xscale=1.2] 
  \begin{scope}
    \foreach \i in {0,...,1} {
        \foreach \x/\y/\n in {1/1/b, 2/2/bc, 3/1/c}
            \node[vertex] (\n\i) at (\i*4+\x,\y) { \n};
        \node[vertex] (e\i) at (\i*4+2,0) {$\varepsilon$};
    }
    
    \foreach \y in {0,...,2}
      \node[draw=none] at (0,\y) {\y};
    
    \draw [-] (3.5,1.5) -- (3.56,1.5);
    \draw [->,decorate,decoration={snake, post length=0.4mm}] (3.56,1) -- (4.5,1);
    
    \path (b0) edge[->, bend left=15] (bc0);
    \path (b0) edge[->, bend right=15] (bc0);
    \path (bc0) edge[->] (c0);

    \path (b1) edge[->, bend left=15] (bc1);
    \path (b1) edge[->, dashed, bend right=15] (bc1);
    \path (bc1) edge[->, dashed] (c1);
    \path (b1) edge[->] (e1);
    \path (e1) edge[->] (c1);
  \end{scope}
\end{tikzpicture}
\end{center}

\caption{Correctness of the collapse in case when the edge \sedge{bc}{c} is present.}\label{fig:lvl1lemmabcc}
\end{figure}

Hence, there is at least one edge \sedge{bc}{bcd}. Then $ {\tt bcd} \in {\cal S} $, and then initially there were two pairs of edges \sedge{bc}{bcd} and \sedge{bcd}{cd} in the graph. So, from {\tt bcd}, at least one collapse was made, and it led to the appearance of the edge \sedge{bc}{c}. But at the current iteration the edge \sedge{bc}{c} can't exist, so it was also collapsed. It's a contradiction with the lemma \ref{lem:bccsave}.
\end{proof}

\begin{figure}[ht]
\begin{center}

\begin{tikzpicture}[xscale=1.2] 
  \begin{scope}
    \foreach \i in {0,...,2} {
        \foreach \x/\y/\n in {1/1/b, 2/2/bc, 3/1/c, 3/3/bcd}
            \node[vertex] (\n\i) at (\i*4+\x,\y) { \n};
        \node[vertex] (e\i) at (\i*4+2,0) {$\varepsilon$};
    }
    
    \foreach \y in {0,...,3}
      \node[draw=none] at (0,\y) {\y};
    
    \draw [-] (3.5,1.5) -- (3.56,1.5);
    \draw [->,decorate,decoration={snake, post length=0.4mm}] (3.56,1.5) -- (4.5,1.5);
    
    \draw [-] (7.5,1.5) -- (7.56,1.5);
    \draw [->,decorate,decoration={snake, post length=0.4mm}] (7.56,1.5) -- (8.5,1.5);
    
    \path (b0) edge[->, bend left=15] (bc0);
    \path (b0) edge[->, bend right=15] (bc0);
    \draw[->] (bc0) to node [sloped] {\(\not\quad\)} (c0);
    
    \path (b1) edge[->, bend left=15] (bc1);
    \path (b1) edge[->, bend right=15] (bc1);
    \draw[->] (bc1) to node [sloped] {\(\not\quad\)} (c1);
    \path (bc1) edge[->, bend left=15] (bcd1);
    \path (bc1) edge[->, bend right=15] (bcd1);
    
    \path (b2) edge[->, bend left=15] (bc2);
    \path (b2) edge[->, bend right=15] (bc2);
    \path (bc2) edge[->, ultra thick] (c2);
    \path (bc2) edge[->, bend left=15] (bcd2);
    \path (bc2) edge[->, bend right=15, dashed] (bcd2);
  \end{scope}
\end{tikzpicture}
\end{center}

\caption{The scheme of proof of the lemma \ref{lem:bbcone}.}\label{fig:lvl1lemmabbcone}
\end{figure}

So, there is only one edge \sedge{b}{bc}.

\begin{lemma}
\label{lem:bcup}
The vertex $ {\tt bc} $ is weakly connected with at least one vertex at the level $ 3 $.
\end{lemma}
\begin{proof}
Let it not be so. Then $ {\tt bc} $ can contain only edges \sedge{b}{bc}, and only edges \sedge{bc}{c} can come from it. By the lemma \ref{lem:bbcone}, there is only one instance of the edge \sedge{b}{bc}. Hence, there is exactly one instance of the edge \sedge{bc}{c}. But then the algorithm had to collapse this pair of edges when examining $ {\tt bc} $, because this operation would only separate $ {\tt bc} $ from all other vertices without violating the connectivity of $ \varepsilon $ and $ {\cal S} $. Contradiction.
\end{proof}

By lemma \ref{lem:bcup} either there exists $ {\tt d} $ such that there is an edge \sedge{bc}{bcd}, or there is a symbol $ {\tt a} $ such that there is an edge \sedge{abc}{bc}. Suppose there is no such $ {\tt a} $, which means that there exists some $ {\tt d} $.

Then $ {\tt bcd} \in {\cal S} $, and therefore initially there were two instances of edges \sedge{bc}{bcd} and \sedge{bcd}{cd} in the graph. So, at least one collapse was made from {\tt bcd}, and it led to the appearance of the edge \sedge{bc}{c}. The edge \sedge{bc}{c} could not be collapsed by the lemma \ref{lem:bccsave}. Hence, some other edge must come to $ {\tt bc} $, and by lemma \ref{lem:bbcone} it can not be another instance of \sedge{b}{bc}.

Hence, there is an $ {\tt a} $ such that there is an edge \sedge{abc}{bc}. Then $ {\tt abc} \in {\cal S} $, and from $ {\tt abc} $ exactly one collapse was made, which led to the appearance of two edges \sedge{ab}{b} and \sedge{b}{bc}, which were not collapsed at the current moment by the lemma \ref{lem:bccsave}.

Therefore, $ {\tt bc} $ contains at least two edges: \sedge{abc}{bc} and \sedge{b}{bc}. Hence, at least two edges come from $ {\tt bc} $.

\begin{lemma}
\label{lem:bccexist}
At the current iteration, there is at least one edge \sedge{bc}{c}.
\end{lemma}
\begin{proof}
Let it not be so. Then, since at least one edge must come from $ {\tt bc} $, it is an edge of the form \sedge{bc}{bcd}. So, $ {\tt bcd} \in {\cal S} $, initially there were two instances of edges \sedge{bc}{bcd} and \sedge{bcd}{cd} in the graph, and there was made at least one collapse from {\tt bcd}, which led to the appearance of the edge \sedge{bc}{c}. Contradiction.
\end{proof}

So, by the lemma \ref{lem:bbcone}, there is one edge \sedge{b}{bc} in the graph, and by the \ref{lem:bccexist} lemma there is at least one edge \sedge{bc}{c}.

But we assert that this pair of edges should have been collapsed, which will contradict the lemma \ref{lem:bccsave} and the definition of the algorithm which makes the maximum possible number of collapses.

The point is that even after the collapse there will be weak paths $ {\tt b} - \varepsilon - {\tt c} $ and $ {\tt b} - {\tt ab} - {\tt abc} - {\tt bc} $, and hence the connectivity of the vertices $ {\tt b} $, $ {\tt c} $ and $ {\tt bc} $ will be preserved after the collapse (see figure \ref{fig:lvl1final}).
\begin{figure}[ht]
\begin{center}

\begin{tikzpicture}[xscale=1.3] 
  \begin{scope}
    \foreach \x/\y/\n in {1/2/ab, 2/1/b, 2/3/abc, 3/2/bc, 4/1/c}
      \node[vertex] (\n) at (\x,\y) { \n};

    \node[vertex] (e) at (3,0) {$\varepsilon$};
    
    \foreach \y in {0,...,3}
      \node[draw=none] at (0,\y) {\y};
    
    \foreach \s/\t in {ab/b, ab/abc, abc/bc}
      \path (\s) edge[->] (\t);
    
    \draw[->] (b) to node [sloped] {\(\not\quad\)} (bc);
    \draw[->] (bc) to node [sloped] {\(\not\quad\)} (c);
    
    \path (b) edge[->, dashed] (e);
    \path (e) edge[->, dashed] (c);
  \end{scope}
\end{tikzpicture}
\end{center}

\caption{The final state of the graph.}\label{fig:lvl1final}
\end{figure}

In this case, one more correct collapse can be made from $ {\tt bc} $. Contradiction.

This completes the proof of the theorem for the $3$-SCS instance.