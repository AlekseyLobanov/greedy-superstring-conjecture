\section{Further Directions and Open Problems}
The most immediate open problem is to prove the Collapsing Conjecture or the Weak Greedy Hierarchical Conjecture.
It would also be interesting to find other applications of the 
hierarchical graphs. We list two such potential applications below.
\begin{description}
\item[Exact algorithms.] Can one use hierarchical graphs to solve SCS exactly in time $(2-\varepsilon)^n$?
It was shown in Section~\ref{sec:intro} that the SCS problem is a special case of the Traveling Salesman Problem. The best known exact algorithms for Traveling Salesman run in time $2^n \poly(|\inp|)$~\cite{B1962, HK1971, KGK1977, K1982, BF1996}. These algorithms stay the best known for the SCS problem as well. The hierarchical graphs were introduced~\cite{scs_exact} for an algorithm solving SCS on strings of length at most $r$ in time $(2-\varepsilon)^n$ (where $\varepsilon$ depends only on $r$). Can one use the hierarchical graph to solve exactly the general case of SCS in time $(2-\varepsilon)^n$ for a constant $\varepsilon$?

\item[Genome assembly.] The hierarchical graph in a~sense
generalizes de Bruijn graph. The latter one is heavily used
in genome assembly~\cite{pevzner2001eulerian}.
Can one adopt the hierarchical graph for this task? For this, one
would need to come up with a~compact representation of the graph
(as datasets in genome assembly are massive) as well as with a~way of
handling errors in the input data. Cazaux and Rivals~\cite{cazaux2018hierarchical} propose a linear-space counterpart of the hierarchical graph.

\item[Optimal solution and a cycle cover.] A superstring corresponds to a Hamiltonian path in the overlap graph, thus, a minimum-weight cycle cover gives a natural lower bound on its length. 
The Greedy Conjecture claims that a greedy solution never exceeds twice the length of an optimal solution. It is also  believed (see, e.g., \cite{weinard2006greedy,laube2005conditional}) that the greedy solution does not exceed the length of an optimal solution plus the length of an optimal cycle cover. This suggests an analogous form of the Collapsing Conjecture: start with any solution and a cycle cover, and then apply normalization, this should result in the same set of arcs regardless of what initial solution and cycle cover one started with. We tested this conjecture and did not find any counter-examples. We believe that exploring this conjecture and its connections to other conjectures would be an interesting further direction of research.
\end{description}
