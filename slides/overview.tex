\section{Superstring Problem: Overview}


\begin{frame}{SCS}
\myproblem{Shortest common superstring problem}{A~set $\{s_1, \dots, s_n\}$ of $n$ strings.}{A~shortest string containing each $s_i$ as a substring.}
\pause
Practical applications: data
storage, data compression, genome assembly
\end{frame}

%\begin{frame}{Overlap}
%\begin{mypic}
%%\draw[help lines] (0,-3) grid (10,3);
%\useasboundingbox (-1,-3) rectangle (11,4);
%\draw[line width=2pt] (0,2.5) -- (6,2.5);
%\draw[line width=2pt] (10,1.5) -- (2,1.5);
%
%\node[rectangle,anchor=east] at (0,2.5)  {$s$};
%\node[rectangle,anchor=west] at (10,1.5) {$t$};
%
%\foreach \x in {2, 6}
%  \draw[dashed,gray,line width=1pt] (\x,3) -- (\x,-0.5);
%
%\foreach \f/\s/\y/\t in {
%  %0/3/0.5/{$\text{prefix}(s,t)$}, 
%  %6/10/0.5/{$\text{suffix}(s,t)$}, 
%  2/6/0/{$\text{overlap}(s,t)$}}
%\path[<->,draw,line width=1pt] (\f,\y) -- node[fill=white,inner sep=1pt,rectangle] {\t} (\s,\y);
%
%\node<2->[rectangle,text width=70mm] at (4.5, -2) 
%{$\text{overlap}(\text{DAB}, \text{ABE})=2$\\
% $\text{overlap}(\text{DAB}, \text{ADE})=0$};
%\end{mypic}
%\end{frame}
%
%\begin{frame}{SCS: permutation problem}
%\begin{mypic}
%\begin{scope}[scale=0.9]
%   \useasboundingbox (0,0) rectangle (12,7);
%   
%   \node<3-> (s) at (6,2) {\bf DABECACBDFA};
%   
%   \node<4-> at (6,0) {Thus, can be solved in time $n!$};
%   
%   \foreach \s/\x/\z in {ABE/1/3, DFA/3/11, DAB/5/1, CBD/9/9, ECA/7/5, ACB/11/7} {
%     \node[rectangle,inner sep=.5mm] (\s) at (\x,6) {\bf \s}; 
%     \node<2->[rectangle,inner sep=.5mm] (a\s) at (\z,4) {\bf \s}; 
%     \path<2-> (\s) edge[-,mc] (a\s);
%     
%     \path<3-> (a\s) edge[->,mc] (s);
%   }
%\end{scope}
%\end{mypic}
%\end{frame}
%
%\begin{frame}[label=complete]{Overlap graph: SCS$\to$MAX-ATSP}
%  %CDEB, BCDE, DBCD, CDBC, BCDB
%
%  \begin{mypic}
%  \begin{scope}[scale=0.9]
%    \useasboundingbox (0,-1) rectangle (12,7);
%    %draw[help lines] (0,0) grid (16,9);
%    
%    \node<1-> (top) at (6,7) {\bf ABE DFA DAB CBD ECA ACB};
%    
%    
%    
%	\begin{scope}[xshift=6cm,yshift=3cm]
%	  \foreach \n/\a in {DFA/0, DAB/60, ABE/120, CBD/180, ECA/240, ACB/300}
%	    \node<2->[draw,rectangle,inner sep=.5mm] (\n) at (\a:30mm) {\bf \n};
%	\end{scope}
%	
%	%\pause
%	
%	\foreach \s/\t/\w in {ACB/CBD/2,  ACB/DAB/0,  ACB/ECA/0,  ACB/DFA/0,  ACB/ABE/0,  CBD/ACB/0,  CBD/DAB/1,  CBD/ECA/0,  CBD/DFA/1,  CBD/ABE/0,  DAB/ACB/0,  DAB/CBD/0,  DAB/ECA/0,  DAB/DFA/0,  DAB/ABE/2,  ECA/ACB/1,  ECA/CBD/0,  ECA/DAB/0,  ECA/DFA/0,  ECA/ABE/1,  DFA/ACB/1,  DFA/CBD/0,  DFA/DAB/0,  DFA/ECA/0,  DFA/ABE/1,  ABE/ACB/0,  ABE/CBD/0,  ABE/DAB/0,  ABE/ECA/1,  ABE/DFA/0
%	}
%		\path<3-5>[->,draw=gray] (\s) edge[bend left=10] node[near start,fill=white,rectangle,inner sep=0.5mm] {\small \textcolor{gray}{\w}} (\t);
%		
%	\foreach \s/\t/\w in {ACB/CBD/2, DAB/ACB/0, CBD/DAB/1}
%		\path<4>[->,draw=mc,line width=3pt] (\s) edge[bend left=10] node[near start,fill=white,rectangle,inner sep=0.5mm] {\bf \alert{\w}} (\t);
%		
%		
%	\foreach \s/\t/\w in {DAB/ABE/2, ABE/ECA/1, ECA/ACB/1, ACB/CBD/2, CBD/DFA/1}
%		\path<6->[->,draw=mc,line width=3pt] (\s) edge node[near start,fill=white,rectangle,inner sep=0.5mm] {\bf \alert{\w}} (\t);	
%	
%	%\node<7> at (6,-1) {\bf DAB ABE ECA ACB CBD DFA $\to$ DABECACBDFA};
%	\node<7> (s) at (6,-1) {\bf \alert{DABECACBDFA}};
%	
%	\node<8> (s) at (6,-1) {Hence, can be solved in $2^n$};
%	
%	%\path<7-> (top) edge[out=180,in=180,gray,->] (CBD);
%	%\path<7-> (DFA) edge[out=0,in=0,gray,->] (s);
%	\end{scope}
%  \end{mypic}
%\end{frame}

\begin{frame}{Known approximation algorithms}
\small
\begin{tabular}{lll}
$3.000$ & Blum, Jiang, Li, Tromp and Yannakakis~ & 1991\\
$2.889$ & Teng, Yao & 1993\\
$2.834$ & Czumaj, Gasieniec, Piotrow, Rytter~ & 1994\\
$2.794$ & Kosaraju, Park, Stein & 1994\\
$2.750$ & Armen, Stein & 1994\\
$2.725$ & Armen, Stein & 1995\\
$2.667$ & Armen, Stein & 1996\\
$2.596$ & Breslauer, Jiang, Jiang & 1997\\
$2.500$ & Sweedyk & 1999\\
$2.500$ & Kaplan, Lewenstein, Shafrir, Sviridenko & 2005\\
$2.500$ & Paluch, Elbassioni, van~Zuylen~ & 2012\\
$2.479$ & Mucha & 2013\\
\pause
\alert{$2.000$} & \alert{greedy?} & \alert{????}
\end{tabular}
\end{frame}

\begin{frame}{Known inapproximability results}
\begin{center}
\begin{tabular}{lll}
$1.00006$ & Ott & 1999\\
$1.00083$ & Vassilevska & 2005\\
$1.00302$ & {Karpinski, Schmied} & 2012\\
\end{tabular}
\end{center}
\end{frame}

\begin{frame}{Greedy algorithm}
\begin{itemize}[<+->]
\item While there is more than one string, take two strings with maximum overlap and replace them with their superstring
\item Greedy conjecture: the greedy algorithm is 2-approximate [Storer 1987]
\end{itemize}
\end{frame}

\begin{frame}{Tight example}
%\begin{align*}
%s_1&={\tt cc}({\tt ae})^n\\
%s_2&= ({\tt ea})^{n+1}\\
%s_3&= ({\tt ae})^n{\tt cc}
%\end{align*}
\begin{itemize}[<+->]
\item ${\tt cc}({\tt ae})^n$, $({\tt ea})^{n+1}$, $ ({\tt ae})^n{\tt cc}$
\item The optimal superstring ${\tt cc}({\tt ae})^{n+2}{\tt cc}$ has length $2n+8$
\item The greedy algorithm produces a superstring ${\tt cc}({\tt ae})^n{\tt cc}({\tt ea})^{n+1}$ of length $4n+6$
\end{itemize}
\end{frame}


%\begin{frame}[label=greedy]{Greedy conjecture}
%  \begin{center}
%  \begin{tikzpicture}[line width=2pt,scale=0.9]
%    \useasboundingbox (0,-1) rectangle (12,7);
%    %draw[help lines] (0,0) grid (16,9);
%    
%    \node<1-> (top) at (6,7) {\bf CABABAB ABABABD BABABA};
%    
%    
%    
%	\begin{scope}[xshift=6cm,yshift=35mm]
%	  \foreach \n/\a in {CABABAB/0, ABABABD/120, BABABA/240}
%	    \node<2->[draw,rectangle,inner sep=1mm] (\n) at (\a:25mm) {\bf \n};
%	\end{scope}
%	
%	%\pause
%	
%	\foreach \s/\t/\w/\l in {CABABAB/ABABABD/6/0, ABABABD/BABABA/0/5, BABABA/CABABAB/0/5} {
%		\path<3->[->,draw=gray] (\s) edge[bend left=10] node[near start,fill=white,rectangle,inner sep=0.5mm] {\small \textcolor{gray}{\w}} (\t);
%		\path<3->[->,draw=gray] (\t) edge[bend left=10] node[near start,fill=white,rectangle,inner sep=0.5mm] {\small \textcolor{gray}{\l}} (\s);
%	}
%	
%	
%	  	\foreach \s/\t/\w in {CABABAB/ABABABD/6, ABABABD/BABABA/0}
%		\path<4>[->,draw=mc] (\s) edge[bend left=10,line width=3pt] node[near start,fill=white,rectangle,inner sep=0.5mm] {\small \alert{\w}} (\t);
%		\node<4->[above right, text width=120mm] at (0,-0.5) {\alert{\bf GREEDY: CABABABDBABABA (14)}};
%	
%
%	\only<5->{
%	  	\foreach \s/\t/\w in {CABABAB/BABABA/5, BABABA/ABABABD/5}
%		\path[->,draw=mc] (\s) edge[bend left=10,line width=3pt] node[near start,fill=white,rectangle,inner sep=0.5mm] {\small \alert{\w}} (\t);
%		\node[above right, text width=120mm] at (0,-1.5) {\alert{\bf OPT: CABABABABD (10)}};
%	}
%	
%	\node<6->[fill=myred!20,rounded corners,above right,text width=100mm] at (0,-1.5) {Greedy conjecture: a~superstring resulting from such a~greedy process is at most two times longer than an optimal superstring};
%		
%
%  \end{tikzpicture}
%  \end{center}
%\end{frame}

%\begin{frame}{Greedy Algorithm}
%\small
%%\begin{algorithm}[ht]
%%\label{algo:ga}
%%\caption{Greedy Algorithm (GA)}
%%\hspace*{\algorithmicindent} \textbf{Input:} set of strings~${\cal S}$.\\
%%\hspace*{\algorithmicindent} \textbf{Output:} a~superstring~for $\mathcal{S}$.
%\begin{algorithmic}[1]
%\While{$\mathcal{S}$ contains at least two strings}
%\State extract from $\mathcal{S}$ two strings with the maximum overlap
%\State add to $\mathcal{S}$ the shortest superstring of these two strings
%\EndWhile
%\State return the only string from $\mathcal{S}$
%\end{algorithmic}
%%\end{algorithm}
%\end{frame}


%\begin{frame}{De Bruijn graph}
%  \begin{center}
%  \begin{tikzpicture}[line width=2pt,scale=1.1]
%    \useasboundingbox (0,0) rectangle (12,7);
%    %\draw[help lines] (0,0) grid (12,7);
%      
%      \node<1-> at (6,7) {\bf GACG CGTA ACGT ACGA CGAG};     
%      
%      \foreach \x/\y/\n in {5/5/ACG, 4/3/CGA, 6/3/GAC, 7/5/CGT, 8/3/GTA}
%        \node<2->[draw,rectangle,inner sep=1mm] (\n) at (\x,\y) {\bf \n};
%
%      \foreach \s/\t in {ACG/CGA, CGA/GAC, GAC/ACG, ACG/CGT, CGT/GTA}
%        \path<2-> (\s) edge[->]  (\t);
%        
%      \only<3>{\draw[gray,->] (3,6.5) to [out=-90,in=180] (5.4,4);}
%    
%      \draw<4->[mc, line width=4pt,rounded corners=8pt,->,xshift=2cm]
%        (2.75,4.5) -- (2.5,3.5) -- (3.5,3.5) -- (3,4.5) -- (5,4.5) -- (5.5,3.5);
%        
%      \node<5-> at (6,1) {\bf \alert{ACGACGTGA}};
%      
%      \draw<6->[gray,->] (3,6.5) to [out=-90,in=180] (5.4,4);
%      \draw<6->[gray,->] (5.4,4) to [out=180,in=90] (5.8,1.4);
%      \draw<6->[gray,-] (5.2,1.3) -- (6.4,1.3);
%  \end{tikzpicture}
%  \end{center}
%\end{frame}
%
%\begin{frame}{SCS: Best Known Algorithms}
%\begin{itemize}[<+->]
%\item Best known exact algorithm: $2^n$
%\item Best known approximation algorithm: $2\frac{11}{23}$ [Mucha 2016]
%\item Greedy conjecture
%\item De Bruijn graphs can be used to solve SCS in polynomial time in two special cases:\\
%\begin{itemize}[<+->]
%\item if $\{s_1, \dotsc, s_n\}$ is a $k$-spectrum of an unknown superstring (i.e., all substrings of length~$k$)
%\item if $|s_i| \le 2$ for all~$i$
%\end{itemize}
%\end{itemize}
%\end{frame}
